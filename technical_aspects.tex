В ходе создания проекта будет использоваться блокировка проема на открытие(1) и проход(2). Под данными терминами подразумевается установление 
самодельных устройств, схожих по принципу работы с магнитноконтактным извещателем(1), датчиком движения(2) и фотоном-ш(2). Они будут извещать 
систему об вторжении на территорию под сигнализацией. 

Попробуем создать свой датчик движения. Начнем с того, что существуют различные типы датчиков:
1 радиоволновые – посылают в заданную область определенную частоту радиоволн, в случае появления препятствия волны отражаются и антенна воспринимает 
это излучение, подавая соответствующий сигнал в ответ;
2 инфракрасные – основаны на принципе изменения состояния электронного ключа под воздействием светового излучения;
3 магнитные – представляют собой аналог кнопки, устанавливаемой на двери или калитке, срабатывают при открытии, такой тип датчика имеет существенные 
ограничения в работе;
4 тепловые – реагируют на появление предметов с определенной температурой в зоне охвата, пригодны для использования в помещениях или после захода солнца;
Мы определили свой выбор на инфракрасном датчике движения, ведь он не имеет значительных минусов и применяется гораздо чаще остальных. 
На рисунке pic1 представлена схема похожего датчика.
Для постройки датчика инфракрасного излучения понадобятся следующие компоненты:

Инфракрасный датчик:

Это может быть пиродетектор, например, MLX90614 или другой аналогичный сенсор, который может обнаруживать инфракрасное излучение.
Микроконтроллер:

Например, Arduino, Raspberry Pi или другой микроконтроллер, который будет обрабатывать сигналы с датчика.
Резисторы и конденсаторы:

Для создания схемы могут понадобиться дополнительные резисторы и конденсаторы для стабилизации сигнала и питания.
Питание:

Блок питания или батареи для обеспечения питания для датчика и микроконтроллера.
Проводки:

Соединительные провода для подключения компонентов друг к другу.
Плата для прототипирования:

Макетная плата (breadboard) для временной сборки вашей схемы.
Специальные модули (по желанию):

Если вы хотите сделать более сложный проект, можно добавить модули передачи данных (например, Bluetooth или Wi-Fi) для передачи информации на другие устройства.
Программное обеспечение:

Средства разработки для программирования микроконтроллера, например, Arduino IDE.
Также может понадобиться дополнительное оборудование, такое как мультиметр, для отладки подключения и проверки работы датчика.

Пример кода датчика (язык си):
#include <iostream>
#include <thread>
#include <chrono>

class DoorSensor {
public:
    DoorSensor() : isOpen(false) {}

    void openDoor() {
        isOpen = true;
        std::cout << "Дверь открыта!" << std::endl;
    }

    void closeDoor() {
        isOpen = false;
        std::cout << "Дверь закрыта!" << std::endl;
    }

    bool isDoorOpen() const {
        return isOpen;
    }

private:
    bool isOpen;
};

class AlarmSystem {
public:
    void activateAlarm() {
        std::cout << "Сигнализация активирована! Звонок тревоги!" << std::endl;
    }

    void deactivateAlarm() {
        std::cout << "Сигнализация деактивирована." << std::endl;
    }
};

int main() {
    DoorSensor doorSensor;
    AlarmSystem alarm;

    // Симуляция работы системы
    while (true) {
        char command;
        std::cout << "Введите 'o' для открытия двери, 'c' для закрытия двери, 'q' для выхода: ";
        std::cin >> command;

        if (command == 'o') {
            doorSensor.openDoor();
            alarm.activateAlarm(); // Активируем сигнализацию при открытии
        } else if (command == 'c') {
            doorSensor.closeDoor();
            alarm.deactivateAlarm(); // Деактивируем сигнализацию при закрытии
        } else if (command == 'q') {
            break; // Выход из программы
        } else {
            std::cout << "Некорректная команда!" << std::endl;
        }

        // Задержка для предотвращения быстрого цикла
        std::this_thread::sleep_for(std::chrono::milliseconds(100));
    }

    return 0;
}
