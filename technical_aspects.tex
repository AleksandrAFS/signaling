В ходе создания проекта будет использоваться блокировка проема на открытие(1) и проход(2). Под данными терминами подразумевается установление 
самодельных устройств, схожих по принципу работы с магнитноконтактным извещателем(1), датчиком движения(2) и фотоном-ш(2). Они будут извещать 
систему об вторжении на территорию под сигнализацией. 

Попробуем создать свой датчик движения. Начнем с того, что существуют различные типы датчиков:
1 радиоволновые – посылают в заданную область определенную частоту радиоволн, в случае появления препятствия волны отражаются и антенна воспринимает 
это излучение, подавая соответствующий сигнал в ответ;
2 инфракрасные – основаны на принципе изменения состояния электронного ключа под воздействием светового излучения;
3 магнитные – представляют собой аналог кнопки, устанавливаемой на двери или калитке, срабатывают при открытии, такой тип датчика имеет существенные 
ограничения в работе;
4 тепловые – реагируют на появление предметов с определенной температурой в зоне охвата, пригодны для использования в помещениях или после захода солнца;
Мы определили свой выбор на инфракрасном датчике движения, ведь он не имеет значительных минусов и применяется гораздо чаще остальных. 
На рисунке pic1 представлена схема похожего датчика.
